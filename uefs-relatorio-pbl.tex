%% abtex2-modelo-trabalho-academico.tex, v-1.9.2 laurocesar
%% Copyright 2012-2014 by abnTeX2 group at http://abntex2.googlecode.com/ 
%%
%% This work may be distributed and/or modified under the
%% conditions of the LaTeX Project Public License, either version 1.3
%% of this license or (at your option) any later version.
%% The latest version of this license is in
%%   http://www.latex-project.org/lppl.txt
%% and version 1.3 or later is part of all distributions of LaTeX
%% version 2005/12/01 or later.
%%
%% This work has the LPPL maintenance status `maintained'.
%% 
%% The Current Maintainer of this work is the abnTeX2 team, led
%% by Lauro César Araujo. Further information are available on 
%% http://abntex2.googlecode.com/
%%
%% This work consists of the files abntex2-modelo-trabalho-academico.tex,
%% abntex2-modelo-include-comandos and abntex2-modelo-references.bib
%%

% ------------------------------------------------------------------------
% ------------------------------------------------------------------------
% abnTeX2: Modelo de Trabalho Academico (tese de doutorado, dissertacao de
% mestrado e trabalhos monograficos em geral) em conformidade com 
% ABNT NBR 14724:2011: Informacao e documentacao - Trabalhos academicos -
% Apresentacao
% ------------------------------------------------------------------------
% ------------------------------------------------------------------------

\documentclass[
	% -- opções da classe memoir --
	11pt,				% tamanho da fonte
	openany,			% capítulos começam em pág ímpar (insere página vazia caso preciso)
	oneside,
	%twoside,			% para impressão em verso e anverso. Oposto a oneside
	a4paper,			% tamanho do papel. 
	% -- opções da classe abntex2 --
	chapter=TITLE,		% títulos de capítulos convertidos em letras maiúsculas
	section=TITLE,		% títulos de seções convertidos em letras maiúsculas
	%subsection=TITLE,	% títulos de subseções convertidos em letras maiúsculas
	%subsubsection=TITLE,% títulos de subsubseções convertidos em letras maiúsculas
	% -- opções do pacote babel --
	english,			% idioma adicional para hifenização
	%french,				% idioma adicional para hifenização
	%spanish,			% idioma adicional para hifenização
	brazil				% o último idioma é o principal do documento
	]{abntex2}

\usepackage{uefs-pbl-report-abntex2}

% ---
% Pacotes básicos 
% ---
\usepackage{lmodern}			% Usa a fonte Latin Modern			
\usepackage[T1]{fontenc}		% Selecao de codigos de fonte.
\usepackage[utf8]{inputenc}		% Codificacao do documento (conversão automática dos acentos)
\usepackage{lastpage}			% Usado pela Ficha catalográfica
\usepackage{indentfirst}		% Indenta o primeiro parágrafo de cada seção.
\usepackage{color}				% Controle das cores
\usepackage{graphicx}			% Inclusão de gráficos
\usepackage{microtype} 			% para melhorias de justificação
% ---
		
% ---
% Pacotes adicionais, usados apenas no âmbito do Modelo Canônico do abnteX2
% ---
\usepackage{lipsum}				% para geração de dummy text
% ---

% ---
% Pacotes de citações
% ---
\usepackage[brazilian,hyperpageref]{backref}	 % Paginas com as citações na bibl
\usepackage[alf]{abntex2cite}	% Citações padrão ABNT

% --- 
% CONFIGURAÇÕES DE PACOTES
% --- 

% ---
% Configurações do pacote backref
% Usado sem a opção hyperpageref de backref
\renewcommand{\backrefpagesname}{Citado na(s) página(s):~}
% Texto padrão antes do número das páginas
\renewcommand{\backref}{}
% Define os textos da citação
\renewcommand*{\backrefalt}[4]{
	\ifcase #1 %
		Nenhuma citação no texto.%
	\or
		Citado na página #2.%
	\else
		Citado #1 vezes nas páginas #2.%
	\fi}%
% ---

% ---
% Informações de dados para CABEÇALHO
% ---
\titulo{Modelo de Relatório de Sessão PBL com \abnTeX}
% TODO: lista com vários autores
\autor{Primeiro Autor \and Segundo Autor \and Quarto Autor \and Quinto Autor \and Sexto Autor \and Sétimo Autor \and Oitavo Autor \and Nono Autor \and Décimo Autor}
\local{Feira de Santana}
\data{2014}
% TODO: mudar para tutor
\tutor{Nome completo do Tutor}
\modulo{TEC498 Projeto de Circuitos Digitais}
\instituicao{%
  \textbf{Universidade Estadual de Feira de Santana}
}
\departamento{Tecnologia}
  
\tipotrabalho{Relatório Técnico de Seção PBL}
% ---


% ---
% Configurações de aparência do PDF final

% alterando o aspecto da cor azul
\definecolor{blue}{RGB}{41,5,195}

% informações do PDF
\makeatletter
\hypersetup{
     	%pagebackref=true,
		pdftitle={\@title}, 
		pdfauthor={\@author},
    	pdfsubject={\imprimirpreambulo},
	    pdfcreator={LaTeX with abnTeX2},
		pdfkeywords={abnt}{latex}{abntex}{abntex2}{trabalho acadêmico}, 
		colorlinks=true,       		% false: boxed links; true: colored links
    	linkcolor=blue,          	% color of internal links
    	citecolor=blue,        		% color of links to bibliography
    	filecolor=magenta,      		% color of file links
		urlcolor=blue,
		bookmarksdepth=4
}
\makeatother
% --- 

% --- 
% Espaçamentos entre linhas e parágrafos 
% --- 

% O tamanho do parágrafo é dado por:
\setlength{\parindent}{1.3cm}

% Controle do espaçamento entre um parágrafo e outro:
\setlength{\parskip}{0.2cm}  % tente também \onelineskip
     
% ---
% compila o indice
% ---
\makeindex
% ---

% ----
% Início do documento
% ----
\begin{document}

% Retira espaço extra obsoleto entre as frases.
\frenchspacing 

% ----------------------------------------------------------
% ELEMENTOS PRÉ-TEXTUAIS
% ----------------------------------------------------------
% \pretextual

% ---
% Capa TODO: Remove CAPA
% ---
%\imprimircapa
% ---

% ---
% Folha de rosto
% (o * indica que haverá a ficha bibliográfica)
% TODO: Remove FOLHA DE ROSTO
% ---
% \imprimirfolhaderosto*
% ---


% ----------------------------------------------------------
% ELEMENTOS TEXTUAIS
% ----------------------------------------------------------
\textual

% ---
% Definifindo estilos de cabecalho e rodape
% ---
\pagestyle{uefs-pbl}
\aliaspagestyle{chapter}{uefs-pbl-capitulo}
% ---

% ---
% Cabecalho
% ---
\imprimircabecalho
% ---

%\setlength\parskip{0.2cm}
\begingroup
\let\clearpage\relax


% ----------------------------------------------------------
% Introdução
% ----------------------------------------------------------
\chapter[Introdução]{Introdução}
% ----------------------------------------------------------
Este documento e seu código-fonte são exemplos de referência de uso da classe
\textsf{abntex2}, do pacote \textsf{abntex2cite} e do estilo \textsf{uefs-pbl-report-abntex2}. 
O documento exemplifica a elaboração de relatórios técnico-científicos produzidos 
em sessões tutoriais PBL para as disciplinas do curso de Engenharia de Computação da
Universidade Estadual de Feira de Santana (UEFS). Este modelo baseia-se na ABNT 
NBR 10719:2011 \emph{Informação e documentação - Relatório técnico e/ou científico - Apresentação}.

Este documento deve ser utilizado como complemento dos manuais do \abnTeX\ 
\cite{abntex2classe,abntex2cite,abntex2cite-alf} e da classe \textsf{memoir}
\cite{memoir}. 

\noindent Equipe da Área de Eletrônica e Sistemas

\noindent João Carlos Nunes Bittencourt

% ----------------------------------------------------------
% Capitulo com exemplos de comandos inseridos de arquivo externo 
% ----------------------------------------------------------

\include{textuais/abntex2-modelo-include-comandos}

% ----------------------------------------------------------
% ELEMENTOS PÓS-TEXTUAIS
% ----------------------------------------------------------
\postextual
% ----------------------------------------------------------
\endgroup

% ----------------------------------------------------------
% Referências bibliográficas
% ----------------------------------------------------------
\bibliography{abntex2-modelo-references}

% ---
% Inicia os apêndices
% ---
\begin{apendicesenv}

% Imprime uma página indicando o início dos apêndices
\partapendices

% ----------------------------------------------------------
\chapter{Quisque libero justo}
% ----------------------------------------------------------

\lipsum[50]

% ----------------------------------------------------------
\chapter{Nullam elementum urna vel imperdiet sodales elit ipsum pharetra ligula
ac pretium ante justo a nulla curabitur tristique arcu eu metus}
% ----------------------------------------------------------
\lipsum[55-57]

\end{apendicesenv}
% ---


% ----------------------------------------------------------
% Anexos
% ----------------------------------------------------------

% ---
% Inicia os anexos
% ---
\begin{anexosenv}

% Imprime uma página indicando o início dos anexos
\partanexos

% ---
\chapter{Morbi ultrices rutrum lorem.}
% ---
\lipsum[30]

% ---
\chapter{Cras non urna sed feugiat cum sociis natoque penatibus et magnis dis
parturient montes nascetur ridiculus mus}
% ---

\lipsum[31]

% ---
\chapter{Fusce facilisis lacinia dui}
% ---

\lipsum[32]

\end{anexosenv}

%---------------------------------------------------------------------
% INDICE REMISSIVO
%---------------------------------------------------------------------
% \phantompart
% \printindex
%---------------------------------------------------------------------

\end{document}
